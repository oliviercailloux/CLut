\RequirePackage[l2tabu, orthodox]{nag}
\documentclass[version=last, pagesize, twoside=off, bibliography=totoc, DIV=calc, fontsize=14pt, a4paper, french, english]{scrartcl}
%need to tell arXiv to process with pdflatex with the following line. Then, "Package hyperref Message: Driver (autodetected): hpdftex." Otherwise, urls don’t break, which messes the bibliography. See how to solve this (using latex?)
\pdfoutput=1
\input{preamble/packages}
\input{preamble/math_basics}
\input{preamble/math_mine}
%Requires package xcolor.
\newcommand{\commentOC}[1]{\textcolor{blue}{\small$\big[$OC: #1$\big]$}}
%Requires package babel and option [french]. According to babel doc, need two braces around \selectlanguage to make the changes really local.
\newcommand{\commentOCf}[1]{\textcolor{blue}{{\small\selectlanguage{french}$\big[$OC : #1$\big]$}}}
\newcommand{\commentYM}[1]{\textcolor{red}{\small$\big[$YM: #1$\big]$}}
\newcommand{\commentYMf}[1]{\textcolor{red}{{\small\selectlanguage{french}$\big[$YM : #1$\big]$}}}

%TODO \crefname{axiom}{axiom}{axioms}%might be needed for workaround bug in cref when defining new theorems?
\crefname{examplex}{example}{examples}% I wonder why this is unnecessary in case of singular

%I find these settings useful in draft mode.
%Which line breaks are chosen: accept worse lines, therefore reducing risk of overfull lines. Default = 200.
	\tolerance=2000
%Accept overfull hbox up to...
	\hfuzz=2cm
%Reduces verbosity about the bad line breaks.
	\hbadness 5000
%Reduces verbosity about the underful vboxes.
	\vbadness=1300

\bibliographystyle{abbrvnat}

%TODO test the brackets.
%\newcommand{\llbracket}{\lbrack\!\lbrack}
%\newcommand{\rrbracket}{\rbrack\!\rbrack}

%2212 Minus Sign
\newunicodechar{−}{\ifmmode{-}\else\textminus\fi}
%These symbols can only be used in math mode, for I found no text equivalent.
%03B3 Greek Small Letter Gamma
\newunicodechar{γ}{\gamma}
%03B4 Greek Small Letter Delta
\newunicodechar{δ}{\delta}
%2115 Double-Struck Capital N
\newunicodechar{ℕ}{\mathbb{N}}
%211D Double-Struck Capital R
\newunicodechar{ℝ}{\mathbb{R}}
%21CF Rightwards Double Arrow with Stroke
\newunicodechar{⇏}{\nRightarrow}
%21D2 Rightwards Double Arrow
\newunicodechar{⇒}{\Rightarrow}
%21D4 Left Right Double Arrow
\newunicodechar{⇔}{\Leftrightarrow}
%2227 Logical And
\newunicodechar{∧}{\land}
%2228 Logical Or
\newunicodechar{∨}{\lor}
%2229 Intersection
\newunicodechar{∩}{\cap}
%222A Union
\newunicodechar{∪}{\cup}
%2260 Not Equal To
\newunicodechar{≠}{\neq}
%2264 Less-Than or Equal To
\newunicodechar{≤}{\leq}
%2265 Greater-Than or Equal To
\newunicodechar{≥}{\geq}
%227B Succeeds
\newunicodechar{≻}{\succ}
%2281 Does Not Succeed
\newunicodechar{⊁}{\nsucc}
%22EB Does Not Contain As Normal Subgroup
\newunicodechar{⋫}{\ntriangleright}
%25A1 White Square
\newunicodechar{□}{\Box}
%25B7 White Right-Pointing Triangle
%TODO test \rhd; \ntrianglerighteq from amssymb?
\newunicodechar{▷}{\triangleright}
%27E6 Mathematical Left White Square Bracket – there’s also \llbracket from stmaryrd
\newunicodechar{⟦}{\text{\textlbrackdbl}}
%27E7 Mathematical Right White Square Bracket – there’s also \rrbracket from stmaryrd
\newunicodechar{⟧}{\text{\textrbrackdbl}}
%27FC Long Rightwards Arrow from Bar
\newunicodechar{⟼}{\longmapsto}
%2AB0 Succeeds Above Single-Line Equals Sign
\newunicodechar{⪰}{\succeq}
%301A Left White Square Bracket
\newunicodechar{〚}{\textlbrackdbl}
%301B Right White Square Bracket
\newunicodechar{〛}{\textrbrackdbl}
%→ is defined by default as \textrightarrow, which is invalid in math mode. Same thing for the three other commands. I redefine those four using \DeclareUnicodeCharacter instead of \newunicodechar because the latter warns about the previous definition.
%→ Rightwards Arrow
\DeclareUnicodeCharacter{2192}{\ifmmode\rightarrow\else\textrightarrow\fi}
%¬ Not Sign
\DeclareUnicodeCharacter{00AC}{\ifmmode\lnot\else\textlnot\fi}
%… Horizontal Ellipsis
\DeclareUnicodeCharacter{2026}{\ifmmode\dots\else\textellipsis\fi}
%× Multiplication Sign
\DeclareUnicodeCharacter{00D7}{\ifmmode\times\else\texttimes\fi}


\input{preamble/draw}
\DeclareAcronym{AMCD}{short=amcd, long={Aide Multicritère à la Décision}}
\DeclareAcronym{AR}{short=ar, long={Argumentative Recommender}}
\DeclareAcronym{DA}{short=da, long={Decision Analysis}}
\DeclareAcronym{DM}{short=dm, long={Decision Maker}}
\DeclareAcronym{DP}{short=dp, long={Deliberated Preference}}
\DeclareAcronym{MAVT}{short=mavt, long={Multiple Attribute Value Theory}}
\DeclareAcronym{MCDA}{short=mcda, long={Multicriteria Decision Aid}}
\DeclareAcronym{MIP}{short=mip, long={Mixed Integer Program}}



%\setbeamertemplate{headline}[singleline]

\begin{document}
\title{%
	Learning value functions collaboratively%
}
\author{Florian Yger}
\author{Olivier Cailloux}
\affil{Université Paris-Dauphine, PSL Research University, CNRS, LAMSADE, 75016 PARIS, FRANCE\\
	\href{mailto:olivier.cailloux@dauphine.fr}{olivier.cailloux@dauphine.fr}
}
\makeatletter
	\hypersetup{
		pdfsubject={collaborative learning},
		pdfkeywords={machine learning},
		pdfauthor={Olivier Cailloux, Florian Yger}
	}
\makeatother
\maketitle

\section{Overview}
Assume we have a set of users represented by $I$ of size $m \in \N$, a set of items represented by $J$ of size $n \in \N$, a set of observed pairs $O \subseteq I × J$ and a relation $r$ mapping those observed pairs $(i, j) \in O$ to some rating $r_{ij}$, an integer in $\llbracket{}1, 5\rrbracket$ (throughout the article, this notation designates intervals in the integers). We also have descriptions of the users in a vector space $A$ of dimension $d_A \in \N$, and of the items in a vector space $B$ of dimension $d_B \in \N$: user $i$ is described by $a_i \in A$, and item $j$ is described by $b_j \in B$.

In classical collaborative learning, we wish to obtain compressed descriptions of the users and the items, $\{u_i, i \in I\}$ and $\{v_j, j \in J\}$, with all $u_i$ and $v_j$ being vectors of some size $k \in \N$. We write $U$ the matrix of size $(m, k)$ that has all $u_i^T$ as rows, and $V$ the matrix of size $(k, n)$ that has all $v_j$ as columns. The compressed descriptions should be such that $r_{ij} \approx u_i^T v_j$.

However, we also wish that our compressed descriptions $u_i$ be in a simple relationship with the provided descriptions $a_i$, thus, we search for $u_i \approx f(a_i)$, and similarly, for $v_j \approx g(b_j)$, with $f$ and $g$ “simple”. More precisely, we search for a matrix $W_A$ of size $(k, d_A)$ such that $W_A a_i \approx u_i$, and $W_B$ of size $(k, d_B)$ such that $W_B b_j \approx v_j$. We call $W_A a_i$ and $W_B b_j$ our explainable representation of the users and the items.

We could define $u_i = W_A a_i$ and $v_j = W_B b_j$ and obtain $r_{ij} \approx a_i^T (W_A^T W_B) b_j$, thus we could optimize by searching for the best $(W_A^T W_B)$. But as this could be too constrained, we will also try to relax these constraints and search for good representation that takes into account that $u_i$ should be close to $W_A a_i$, but not necessarily equal.

We assume we are given loss functions $L$, that compare the ratings $r_{ij}$ to our approximations of the ratings, for $(i, j) \in O$; $L_I$, that compare the vectors $u_i$ to our explainable representation of them; and $L_J$, that compare the vectors $v_j$ to our explainable representation of them.

We want to find matrices $U, V, W_A, W_B$ that optimize the following objective:
\begin{equation}
\min_{U, V, W_A, W_B} \sum_{(i, j) \in O} L(r_{ij}, u_i^T v_j) + \sum_{i \in I} L_I(u_i, W_A a_i) + \sum_{j \in J} L_J(v_j, W_B b_j).
\end{equation}

\subsection{Explainable dimensions}
Here we explain that the $W_A$ stuff may be used to determine only a subset $k' < k$ of dimensions, and compare to the above approach with relaxation over the $k$ dimensions.

\section{Data set}
We experiment using the Sushi data set (ref…)

Here $J$ is a set of $100$ sushi types \footnote{$X^*=X_B$ in the original notation.}. The original authors collected the sushi from the menu of 25 restaurants, computed the frequency of each sushi type occurring on a menu, and took the $100$ most frequent. Write $P_J$ the frequency (thus a multiple of $1/25$).

Write $J_\text{top} \subset J$ the $10$ most frequent sushi types \footnote{$X_A$}.

The set $I$ has cardinal $m = 5000$. The authors drew randomly a set of 10 sushi types among $J$, for each user $i$, according to the distribution $P_J$. Let us denote this by $J_i \subset J$ \footnote{$X^B_i$}.

The authors obtained, for each user $i$ in $I$, three preferential informations. First, they collected a preference ordering over the sushi types from $J_\text{top}$ from most to least preferred, with no tie allowed. Let $>^\text{top}_i$ denote that preference ordering \footnote{$O^A_i$} (“top” designates the fact that the ordering is over the 10 most frequent sushi types). Second, they collected a rating, in $\llbracket 1, 5\rrbracket$, for each sushi type in $J_i$. Define $O$ as the set of pairs $(i, j)$ such that $j \in J_i$, and let $r_{ij}$ denote that rating. Third, after two supplementary questions asking how oily the user think each sushi type is and how frequently they eat them, they collected a preference ordering over $J_i$, that we write $>^\text{indiv}_i$ \footnote{$O^B_i$} (“indiv” designates the fact that this relation orders a set of sushi that has been drawn specifically for that user).

Users are described in a space $A$ comprising the attributes $\{$gender, age category, origin-prefecture, origin-region, origin-east/west (binary), current-region, current-east/west (binary), origin-prefecture$\}$. Items are described in a space $B$ comprising the attributes $\{$group, heaviness/oiliness in taste (in $[0, 4]$), frequency of eating (in $[0, 3]$), normalized price (in $[0, 1]$), $P_J(j)$$\}$.

\section{References}
%\bibliography{mybib}

\end{document}

